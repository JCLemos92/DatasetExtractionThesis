\chapter{Conclusion}
\label{chap:Conclusion}

During the development of this study, we tried to tackle the theme of 'Meta Feature classification and insertion on a Synthetic dataset generation process.' With SNOOKER as the original basis, we analyzed the concept of meta features, exploring the subject and presenting the most commonly known family of meta features. While groups of meta features are not uniformly described, the list presentive is extensive and varied enough to characterize a dataset correctly.

The study advanced to dataset generation methodologies. While explored, this subject was not developed further during the development phase.

The Classification feature was appended to SNOOKER. We used the pyfme library to help us extract as many meta features from generated datasets as possible, allowing for vast classification options.

We also explored ways to force specific meta feature results on synthetic datasets. Due to a lack of practical results, our observations were purely theoretical, having no sound proof to correctly evaluate our hypotheses. While we believed that inserting meta features into synthetic datasets is a possible and (in some cases) beneficial endeavour, no testing was finalized.

The development of this study went through more problems than expected. The meta-extraction tool ended up occupying most of the time available for development. There is also an evident lack of knowledge and know-how regarding dataset analysis and generation processes that hindered the development and limited the scope of the study. 

\section{Future Works}
As said before, the study of meta feature inclusion on generation could benefit from further research, being a possible project to be.

We can now classificate datasets using the meta feature classification tool. We could test performance on machine learning models trained using similar datasets with different meta feature values. This is also an interesting topic that requires further research.

Lastly, we present possible advances to SNOOKER as a product. After exploring SNOOKER, it is easy to see that computing power and runtime are essential factors needed when dealing with the generator. We conclude that SNOOKER would profit from a significant overhaul, becoming decentralized. Having all the necessary calculations done remotely on a server. As meta-analysis of some families could take a long time, making them run in a server would prevent users from needing to keep processes running for long times. The results could be sent to the user at a later date. We conclude that SNOOKER would benefit from evolving into a web service.
