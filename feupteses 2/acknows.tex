\chapter*{Agradecimentos \footnote{Unlike the rest of this study, the acknowledgements are written in Portuguese to be read by every person mentioned, some of which are native only speakers.}}

Um trabalho de mestrado é uma longa caminhada, marcada por diversos desafios, incertezas, alegrias e tristezas. Apesar do processo solitário a que qualquer investigador está destinado, o seu percurso é também pautado pelo apoio de várias pessoas, às quais dedico este trabalho. 

Em primeiro lugar, gostaria de exprimir os mais sinceros agradecimentos ao meu orientador, professor doutor Daniel Castro Silva e ao meu co-orientador, Leonardo Silva Ferreira, pela orientação exemplar, rigor científico, disponibilidade e conselhos facultados ao longo dos últimos meses. 

Agradeço aos meus grandes amigos e colegas de curso, Ricardo Figueiredo e Luís Guilherme Neves, por terem sempre puxado por mim e me terem ajudado quando mais precisei – é também graças a vocês que este trabalho é possível. 

Obrigado aos meus amigos Sérgio Pinto, Rahul Pires, João Brandão, João Pedro Morais, Mafalda Neves, Inês Faria e Sara Aguiar por todos os momentos passados e amizade transmitida.  

Um obrigado à Tuna de Engenharia da Universidade do Porto (TEUP) - lugar onde encontrei a minha casa longe de casa, e por ter sido a escola que fez de mim grande parte da pessoa que sou hoje. 

Agradeço à minha terapeuta Dra. Sara Malheiro, pela atenção, disponibilidade, prontidão e amizade que me permitiram trilhar o caminho certo para aqui chegar. 

Ao professor doutor António Augusto Sousa, pelas constantes palavras de encorajamento e por uma conversa especial que fez com que fosse capaz de olhar para este percurso com outros olhos. 

Por fim, e como não podia deixar de ser, dirijo os meus maiores agradecimentos à minha família, que revelou ser o maior porto de abrigo e refúgio nos momentos mais difíceis.  

Aos meus pais, que sempre foram um exemplo de dedicação, trabalho e esforço. 

Ao meu irmão Pedro, meu melhor amigo, com quem partilho memórias e momentos muito felizes.  

Às minhas tias Helena e Elsa, que ao longo da vida (em especial dos anos no Porto) foram sempre a principal rede de apoio e ajuda para tudo, e a quem serei eternamente grato.  

Às minhas tias São, Zeza, e à minha prima Xana pelo carinho e amizade desde sempre.  

À Mariana, um especial e muito grande agradecimento por, mesmo nos momentos mais difíceis, nunca me ter deixado desistir e ter acreditado que tudo isto seria possível. Obrigado pelo carinho, dedicação e amor ao longo dos últimos nove anos.  
\vspace{10mm}
\flushleft{João Carlos Lemos}
