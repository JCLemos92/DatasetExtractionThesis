\chapter*{Abstract}
Machine learning (ML) models can perform a vast amount of tasks. Recognition of images and speech, statistical arbitrage (useful in finance), performing medical diagnostics through the analysis of medical exams, and performing predictive analysis are skills in their repertoire. The ML models must be trained using extensive data to accomplish these tasks.

%Each machine learning model needs to absorb data related to the task it will end up performing, 
Valid data is usually either scarce, expensive, or not readily available. Scrapping tools are commonly used to circumvent the scarceness of data, but their quality is questionable. A small dataset is also insufficient to produce an accurate prediction model. Even in cases where a large amount of data is available, its use may infringe privacy policies (a common problem when entering the medical field). These problems represent an accumulation of problematic data that leads to incorrect results. Synthetic datasets have a proven track record of improving ML effectiveness by providing a better dataset with fewer data imbalances than its "organic" bredren. 

The work developed in this study focus on the study of meta feature extraction methods and meta feature insertion analysis. Using an already existing Ticket-based Synthetic DataSet Generator as a basis, we also developed a meta extractor feature.

Each dataset has its group of meta features - its fingerprint. Our meta feature extractor feature can dissect any kind of CSV tabular dataset, even if those datasets were not synthetically generated by our generator.
%When an existing dataset is available, the generator should analyse it and its meta features extracted, then used as parameters for the generation process. This generation should also be available if no original dataset is provided by making a simple selection of parameters. 
Despite being referenced in many studies, meta features are not uniformly described. Nevertheless, some studies agree on lists of meta features, and those arranged lists are explored. Besides extracting parameters from existing datasets' meta features, it is also essential to analyse the parameter-parameter relationship.
%The quality of the synthetic dataset is also a factor of concern: cases of missing data, imbalances, anomalies and overfitting can easily happen during the generation process, and as such, the number of outliers must be realistic. Outliers should exist in a dataset as they contain useful information.

We also explored the concept of meta feature inclusion in the generation process, where we try to force values of certain meta features into the final synthetic dataset.

%The workflow started with the analysis and definition of a set of meta features, passing to the creation of parameters in the generator corresponding to that same data features. The quality assurance process will be the final step in the practical development of this study.

\vspace*{10mm}\noindent
\textbf{Keywords}: Synthetic Dataset, Meta Features, Machine Learning
